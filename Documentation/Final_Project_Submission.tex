%1) Relation complaint between customer and order and service employee
%2) Relation between order and customer (delivers)

\documentclass[12pt]{report}

\usepackage[utf8]{inputenc}
\usepackage{graphicx}

\graphicspath{..\ER Diagram}

\title{
    \textbf{Database Design for Online Retail Store\\ Winter 2022 \\}
}

\author{Abhimanyu Bhatnagar, 2020273
        \\Atyant Sony, 2020039
        \\Madhava Krishna, 2020217
        \\Ritika Nagar, 2020112}
\date{\textit{\today}}



\begin{document}
    \maketitle
    \tableofcontents
    \pagebreak

    %chapter 1, defining the problem
    \chapter{Defining the Problem}
    \section{Objective}
        The goal is to design a database management system for an online retail store,
        similar to Big Bazaar, Flipkart and Amazon. Our project models on Grofers (now BlinkIt), 
        a fast grocery marketplace for consumers to purchase day-to-day goods from.
    
    \section{Problem Statement}
    The ER diagrams and the relational schema that follow are based on the following (rudimentary) problem statement:
    \\\\
    %Problem statement
    \textit{
        The online retail store serves many customers. 
        The customers are required to hold an account on the platofrm to be able to purchase items.
        They can create an account by specifying their name, email address, phone number, and address.
        Customers add products to a shopping cart. They apply coupons on the shopping cart; the coupons 
        have a coupon code and an associated discount percentage.\\
        Customers order items by checking out the items on their shopping cart. The order is placed once the
         transaction is confirmed. The order is delivered to the customer only, 
         they cannot have the order delivered to other addresses.
        \\
        A product can belong to various categories and has specifications and a cost. Each
        product is obtained from a vendor which the store transacts with. After being purchased from 
        the vendor, the products are stored in a warehouse. Warehouse employees are responsible for packing 
        and preparing orders. The readied order is then delivered to the customer by a delivery agent.
        \\
        In case of any lapses with an order, the customer complains to support staff who create a complaint
        number against the order. They send out the details regarding the complaint to the customer.
    }

    \section{Stakeholders}
    Notable stakeholders of this problem include:
    \begin{enumerate}
        \item Customer
        \item Employees
        \item Suppliers
        \item Management of the company: board of executives, shareholders
    \end{enumerate}
    Other stakeholders include manufacturers, warehousing service providers
    
    \section{Assumptions}
    \begin{enumerate}
        \item The retail store would be operated in one country only. Therefore, there isn't an option to change the country.
        \item There will be a shopping cart associated with the customer's account. They won't be able to add item to cart without being logged in.
        \item The delivery would be taken care of by the company itself; items will be shipped from one warehouse only (Similar to Blinkit, erstwhile Grofers).
        \item Coupons would be applied on the order, not the cart. Coupon would be applied during the checkout process.
        \item Coupons would be applied using the coupon code, which is a unique alphanumeric value. A coupon cannot be reused.
        \item Employees will be divided into delivery partners (responsible for delivering the order), warehouse workers (tasked with preparation of orders) and service employees (responsible for conflict resolution).
        \item Vendors have only one account number and will be based only in India.
        \item Employees have only one email address.
        \item There won't be any wishlisting features, nor any saved-items feature like the ones offered by Amazon.
        \item Employees' performance would be graded on a scale from 1 to 10 (for ease of data entry) with decimal values being permissible.
        \item The mode of payment would be Cash on Delivery (no alternate methods would be provided) for the sake of simplicity.
        \item Coupons are included in the database, but they won't be used (because they added extra complications to the checkout process).
    \end{enumerate}



    \chapter{ER Diagram}
    
    \section{Entities}

    \begin{enumerate}
        \item Customer
            \begin{enumerate}
                \item \underline{Customer ID}: Primary key
                \item Name: composite
                    \begin{enumerate}
                        \item First Name
                        \item Last Name
                    \end{enumerate}
                \item Address: composite
                    \begin{enumerate}
                        \item House Number
                        \item Locality
                        \item City
                        \item State
                        \item Pin Code
                    \end{enumerate}
                \item Email Address
                \item Password
            \end{enumerate}
        \item Product
            \begin{enumerate}
                \item \underline{Product ID}: Primary key
                \item Price
                \item Category
                \item Discount Percentage
                \item GST
            \end{enumerate}
        \item Orders
            \begin{enumerate}
                \item \underline{Order ID}
                \item Total\_Cost
                \item Taxes
                \item Discount Percentage
            \end{enumerate}
        \item Vendor
            \begin{enumerate}
                \item \underline{Vendor ID}
                \item Address
                    \begin{enumerate}
                        \item Plot number
                        \item City
                        \item State
                        \item Pin code
                    \end{enumerate}
                \item Name
                \item Phone number
                \item Email\_address
                \item password
            \end{enumerate}
        \item Coupon
            \begin{enumerate}
                \item \underline{Coupon Code}: Primary key
                \item Discount Percentage
                \item Validity
            \end{enumerate}
        \item Warehouse
            \begin{enumerate}
                \item \underline{Warehouse ID}
                \item Address: composite
                    \begin{enumerate}
                        \item Plot number
                        \item City
                        \item State
                        \item Pin Code
                    \end{enumerate}
                \item Telephone Number: multivalued
            \end{enumerate}
        \item Employee:
            \begin{enumerate}
                \item \underline{Employee ID}: Primary key
                \item Date of Joining
                \item Position
                \item Department
                \item Email Address
                \item Name: composite
                    \begin{enumerate}
                        \item First Name
                        \item Last Name
                    \end{enumerate}
                \item Residential Address: composite
                    \begin{enumerate}
                        \item House number
                        \item Locality
                        \item City
                        \item State
                        \item Pin Code
                    \end{enumerate}
                \item Date of Joining
                \item Performance
                \item Salary
                \item Gender
                \item Date of Birth
                \item Age : derived
                \item Password
            \end{enumerate}
            Employees can be specialised into:
            \begin{itemize}
                \item Delivery Partner:
                    \begin{enumerate}
                        \item Vehicle ID
                        \item Vehicle Type
                    \end{enumerate}
                \item Warehouse Worker: no additional attributes
                \item Service Employee: no additional attributes
            \end{itemize}
        \item Shopping Cart (Weak entity)
            \begin{enumerate}
                \item \underline{Customer ID,Product ID}: Primary Key
                \item Customer ID: Foreign Key (references Customer)
                \item Product ID: Foreign Key (references Product)
                \item Quantity
                \item Total cost: Derived attribute

            \end{enumerate}
        
    \end{enumerate}


    \section{Relations}
        \begin{enumerate}
             \item Supplies: between vendor and product
            \begin{enumerate}
                \item \underline{Vendor ID, Product ID}: Primary Key
                \item Vendor ID: Foreign Key (references Vendor)
                \item Product ID: Foreign Key (references Product)
                \item Quantity: attribute
            \end{enumerate}
            \item Includes: between product and order to show which order contained what products
            \begin{enumerate}
                \item \underline{Order ID, Product ID}: Primary Key 
                \item Order ID: Foreign Key (references Order)
                \item Product ID: Foreign Key (references Product)
                \item Quantity: attribute
            \end{enumerate}
            \item Stores: between warehouse and product to show which warehouse contains what products.
            \begin{enumerate}
                \item \underline{Warehouse ID, Product ID}: Primary Key
                \item Warehouse ID: Foreign Key (references Warehouse)
                \item Product ID: Foreign Key (references Product)
                \item Stocks: attribute
            \end{enumerate}
            \item Delivery: Quarternary relation between warehouse, order, delivery partner and customer to represent delivery.
            \begin{enumerate}
                \item \underline{Order ID}: Primary Key
                \item Order ID: Foreign Key (references Order)
                \item Employee ID: Foreign Key (references Employee)
                \item Customer ID: Foreign Key (references Customer)
                \item Warehouse ID: Foreign Key (references Warehouse)
                \item Delivery\_date
            \end{enumerate}
            \item Transaction: Relation between customer and their cart, order and coupon code.
            \begin{enumerate}
                \item \underline{Order ID}: Primary Key
                \item Customer ID: Foreign Key (references Customer)
                \item Payment Method
                \item Transaction Status
                \item Transaction Time
                \item Coupon Code: Foreign Key (references Coupon)
            \end{enumerate}
            \item Complains: between service employees (specialisation), customer and order to indicate a dispute.
            \begin{enumerate}
                \item \underline {Complaint Number}: Primary Key
                \item Customer ID: Foreign Key (references Customer)
                \item Order ID: Foreign Key (references Order)
                \item Service Employee ID: Foreign Key (references Employee)
                \item date\_of\_crelation
                \item resolved
                \item details
            \end{enumerate}
	\item Works in: between Warehouse Worker and warehouse.
	\begin{enumerate}
				\item \underline{Employee ID, Warehouse ID}: Primary key
			\end{enumerate}
	\item Has: relation between customer and shopping cart (weak entity).
			\begin{enumerate}
				\item \underline{Customer ID}: Primary key
			\end{enumerate}
	\item Contains: relation between shopping cart and product to indicate that shopping cart contains the item.
		\begin{enumerate}
			\item Quantity
			\item Cost of items in the cart: Derived quantity.
		\end{enumerate}
	\end{enumerate}


	\section{Diagram}
	\includegraphics[width =0.85\textwidth]{Pictures/ER_Diagram.png}

\chapter{Relational Schema}
The ER diagram was reduced by noting multiplicities and the following tables for the relational schema resulted.

\begin{enumerate}
	\item Customer(\underline{customer ID}, First\_name, Last\_name, House\_number, Locality, City, Pincode, Email\_Address, Password)
	\item Product(\underline{product ID}, Price, Category, Discount Percentage, GST Percentage)
	\item Vendor(\underline{Vendor ID}, First Name, Last Name, Plot Number, City, Pincode,email\_adress,password)
	\item Warehouse(\underline{Warehouse ID}, Plot Number, City, Pincode)
	\item Employee(\underline{Employee ID}, First name, Last name, Age, Salary, Gender, Department, Performance, Position, Date of Joining, Date of Birth, Email Address, House Number, Locality, City, Pincode, Phone Number,password)
	\item Delivery Partner(\underline{Employee ID}, Vehicle ID, Vehicle Type)
	\item Warehouse Worker(\underline{Employee ID})
	\item Service Employee (\underline{Employee ID})
	\item Orders(\underline{Order ID}, Total Price, Taxes, Total Discount Percentage)
	\item Coupon(\underline{Coupon Code}, Discount Percentage)
	\item Transaction(\underline{Order ID}, Payment Method, Transaction Status ,Transaction Time, Customer ID, Coupon code)\\ Here, Customer ID is referenced from Customer table and coupon code from Coupon table. Coupon code can be null, customer id cannot.
	\item Delivery (\underline{Order ID}, Employee ID, Customer ID, Warehouse ID,delivery\_date) \\ Here Employee ID is taken from the Employee table if it belongs to Delivery Partner, Customer ID from Customer table, Warehouse ID from Warehouse table.
	\item Stores(\underline{Warehouse ID, Product ID}, Quantity)
	\item Supplies(\underline{Vendor ID, Product ID}, Quantity)
	\item Shopping Cart(\underline{Customer ID, Product ID}, Quantity)
	\item Complains(\underline{Complaint Number}, Customer ID, Order ID, Service Employee ID,resolved,date\_of\_creation,details)\\ Here, Customer ID is taken from Customer table, Order ID from Orders table, Service Employee ID from Service Employees table.
	\item Order Products(\underline{Order ID, Product ID}, Quantity)

\end{enumerate}

\textbf{\underline{ Note}}: Employee has both fields of age and DOB. This is a redundancy on our part as the data entries were designed with age as an attribute. This may be eliminated in the final version depending on whether there is need or not (number of employees should be significantly lower than the number of customers).
\\
The SQL queries for creating the database, populating the database and the queries required are submitted under the names: 'database\_creation.sql', 'database\_population.sql' and 'queries.sql' in the Database folder.

\chapter{Changes from the Mid Evaluation}


    \section{Schema}
        \begin{enumerate}
            \item Dropped vendor\_phone, customer\_phone, warehouse\_phone, product\_rating, product\_photos tables to reduce complexity.
            \item Converted complains quarternary relation to ternary relation.
            \item Added attributes email\_address and password in vendor.
            \item Added password attribute in employee table.
            \item Auto-incremented customer, product, vendor, warehouse, orders, complains (for ease of insertion).
            \item Added attribute delivery\_date in delivery.
            \item Removed attribute category from product, added attribute product\_name.
            \item Added attributes details, resolved and date\_of\_creation in complains.
        \end{enumerate}

    \section{Queries and Views}
        \begin{enumerate}
            \item Added another query to account for the missing query during the mid-evaluation.
            \item Created new views and roles.
            \item Created triggers pertaining to the application.
        \end{enumerate}

    
\chapter{Deliverables for Final Evaluation}
    \section{Project Scope and Relational Schema}
        Scope, depth and relation schema were updated from the mid-evaluation. Details in the previous chapter.
    
    \section{Views and Grants}
        Views were created (present in views.sql file in the Database directory). All views were created taking into account the requirement by the application. Corresponding roles were created and were granted authorisation.

    \section{SQL Queries}
        Queries are present in queries\_endsem.sql.\\
        Most of the queries were taken from the embedded queries which were required to run the application. Do note that the data was changed in order to generate the pure SQL queries from embedded queries and hence may not be coherent.

    \section{Embedded SQL Queries}
        All embedded SQL queries (ranging from easy to complex) are in embedded\_sql.txt
    
    \section{Query optimisation}
        Queries are automatically optimised by MySQL, however, we had tried to optimise the number of calls by searching creative ways to only select as many columns as required, and created indexes to assist with querying.

    \section{Indexing}
        MySQL creates indices for primary keys. In addition to those, we had tried to optimise all those functions that we were using in the application. \\
        Attributes for which we created index tables
        \begin{itemize}
            \item email\_address and password from customer
            \item email\_address and password from vendor
            \item stock from stores
            \item quantity from shopping\_cart.
            \item products on shopping\_cart.
            \item product\_id from stores.
            \item customer\_id from transaction
            \item service\_employee\_id from complains
            \item product\_id from supply
        \end{itemize}
    
    \section{Triggers}
        Triggers were created (present in Trigger.sql in the Database directory). The following triggers were created:
        \begin{itemize}
            \item For generating password for employee (uses MD5 checksum to generate temporary queries).
            \item Removing item from cart once it goes out of stock.
            \item Reduce discount percentage once stock reduces below 100.
            \item Insert into stores whenever a new product is inserted (adding it to list before incrementing value).
        \end{itemize}
    
    \textbf{NOTE:} All files pertaining to queries and the submission are in the Database director.

\chapter{Overview of the Sites}

    \textbf{\textit{NOTE: The connector files require to be saved again and again whenever there is a sporadic change in the database. This seems to be an issue with server reload and we were not able to take care of it for this project.
    Example, on changing delivery states in database using MySQL and going into the employee portal to visualise them, the to-be-delivered orders are not visible at all.}}
    \section*{Introduction}
    Keeping in mind that customers and employees and vendors have different requirements, a decision was taken to split the application into two parts: one which took care of the Customers' side, and another which catered to Employees and Vendors.
    \\
    The scope was limited in several places, in order to create a high-quality, coherent product that would work well.
    \\
    \section{Customers' Portal / Main Portal}
    The customers' portal allows for several features:
    
        \begin{enumerate}
            \item Signing up new customers.
            \item Browsing products
            \item Logging in
            \item Adding products to cart
            \item Checking out items
            \item Monitoring orders and delivery dates
            \item Sending complaints regarding orders.
        \end{enumerate}
    A few critical assumptions were made:
    \\
        \begin{itemize}
            \item The payment occurs only through Cash on Delivery. This decision was taken to limit creation of another form for payment details (which were getting hard to handle).
            \item After complaining about an order, the customer would be contacted via email by the assigned service employee, and they would not be able to track the status of their complaints. This decision was made looking at Amazon's flow, wherin complaining results in the company employees contacting the customer.
            \item There is no option to return an order, or rate them, for the same reason that forms were getting overwhelming to handle.
            \item Cart items cannot be dropped as a single unit, the cart can only be fully cleared (in order to reduce complexity).
            \item In case the item stock gets to 0, it would be removed from all customers' carts (due to the trigger).
            \item Delivery details won't be shown.
        \end{itemize}
    

   
    \section{Business Portal}
    This site is used by employees(service employees and delivery guy) and vendor.
        \begin{itemize}
            \item Vendors' portal flow
                \begin{enumerate}
                    \item New Vendor creates his/her account 
                    \item He/She logins to his/her account
                    \item He/she can restock the products he/she is already selling
                    \item He/she can add new products
            \end{enumerate}
            \item Employees' portal flow
                \begin{enumerate}
                    \item Delivery Employees and Service Employees can login to the portal
                    \item Service Employees can see Customers' Complaints and mark them as resolved.
                    \item Delivery guy can see the orders that he/she has to deliver and mark them as delivered.    
            \end{enumerate}
        \end{itemize}
    As with the customers' portal, a few assumptions were made:
    \\
    \begin{itemize}
        \item In employees' portal, we cannot register a new employee (database admin would be responsible for that).
        \item Warehouse workers have no provision to login.
        \item Service employees would only be responsible for handling complaints(would only be marking them as resolved after sending an external email, at least on the portal).
        \item Delivery agents would be allowed to login and they would only be shown a list of deliveries to be made. Only a provision to mark the item as delivered would exist.
        \item We would not cross verify with the customers whether they recieved the order or not. The customer can complain if the order was not recieved.  
        \item Vendors would have an option to add a new product, but would not be shown the product\_id of the newly inserted item (to reduce complexity, they would still be able to visit the customer's site and get to know the product id).
        \item Restocking only happens on the basis of product\_id.
    \end{itemize}

        

        

\chapter{Project Development Procedure and Deliverables}
	\section{ER Formulation}
	Before commencing with the ER diagram formulation, many brainstorming sessions were held to determine which all aspects the online retail store would target and to what degree. The result of those sessions is the \textit{problem statement} and the list of assumptions. The problem statement is a rudimentary sketch of what all scenarios the store should be equipped to handle and what to expect from the usage scenarios of the database management system. The ER diagram and relational schema have been added in the previous chapters and the diagram should be available in the ER\_Diagram folder.
	\section{Weak Entity}
	The shopping cart was determined to be a weak entity since it does not have any identifying attribute other than belonging to a specific customer. Hence, on careful consideration of all its attributes and the lack of a distinctive attribute, it was decided to be kept as a weak entity.
	\section{Entity Relationship Participation and Types}
	This was another topic on which multiple meetings had been conducted, specifically on the nature of the ternary/quarternary relations and how to decompose them. Ultimately, Delivery and Transaction were kept because of their utility and ability to define the whole scenario without conflicts with other entities and relations. The participation types (total and partial), relationship roles and cardinality constraints have been mentioned in the ER diagram. Detailed constraints have been mentioned in the SQL database creation file.

	\section{Ternary Relation}
	A ternary relation: complains was identified which involves an order, the customer who placed that order and one of the service employees. The cardinality is many on the order's and service employee's side and unary on the customer's  side. It was identified in the following way: "A customer can complain to many service employees about many orders."

	\section{Relational Schemas}
	They have been included in one of the earlier chapters.
	
	\section{Sufficient and Valid Constraints in DDL}
	The DDL file has been enclosed in the Database folder under the name 'database\_creation.sql'.

	\section{Data Entry}
	Data was generated using Mockaroo and using python scripts for the foreign key tables. The primary keys were chosen in increasing order (similar to AUTO\_INCREMENT) and most of the tables (except warehouse and complains, which contain 5 and 10 rows respectively), have 50 to over a hundred entries.

	\section{Queries}
	Queries have been included in the Database folder under the file name 'queries.sql', which should be run after running 'database\_creation.sql' and 'data\_population.sql' respectively.
	
\chapter{Contributions of Team Members}
Collaboration was done by the means of Microsoft Visual Studio Code's Live Share feature which allowed for real time simultaneous editing of multiple files by many participants. Communication occurred through Google Meet links. Following are the contribution of each team member in the development and ideation of the project till the mid-evaluation. Many of the responsibilities overlapped and all tasks were completed with contributions from all. 

	\begin{enumerate}
		\item Abhimanyu Bhatnagar:
			\begin{enumerate}
				\item Ideation of the ER diagram
				\item Drawing the ER diagram
				\item Reduction to Relational Schema
				\item DDL : creating tables
				\item Creating test data
				\item Part of the SQL queries
				\item Database connection (embedded SQL queries)
				\item Documentation
				\item Triggers
                \item Views and grants
                \item Indexing             
                \item Helped with the Customer page
            \end{enumerate}


		\item Atyant Sony:
			\begin{enumerate}
				\item Ideation of the ER diagram
				\item Drawing the ER diagram
				\item Reduction to Relational Schema
				\item DDL : creating tables
				\item Creating test data
				\item Part of the SQL queries
				\item Documentation and latex formatting.
				\item Handled HTML forms
                \item Embedded sql queries
                \item Indexing
                \item Frontend for the entirety of Customer's Portal
                \item Responsible for a lot of front end work
                \item Views
			\end{enumerate}
		\item Ritika Nagar:			
            \begin{enumerate}
                \item Part of ideation
                \item Generated data for a few tables
                \item Created product review form (didn't make it into the final application due to complexities involved in linking / POSTing).
                \item Added one SQL query and trigger.
                \item Raised issues pertaining to reduction to relational schema.
            \end{enumerate}
		\item Madhava Krishna:
			\begin{enumerate}
				\item Determining requirements and setting a problem statement.
				\item Drawing the ER diagram
				\item Reduction to Relational Schema
				\item Part of the DDL
				\item Creating part of the test data
				\item SQL queries
				\item Documentation and latex formatting
				\item Views
				\item Database Connection
				\item Customer Portal Landing Page
				\item Part of the employees' portal
				\item Part of the customers' page
                \item Triggers
			\end{enumerate}
	\end{enumerate}
\end{document}